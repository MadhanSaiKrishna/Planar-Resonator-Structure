\documentclass[conference]{IEEEtran}
\IEEEoverridecommandlockouts
% The preceding line is only needed to identify funding in the first footnote. If that is unneeded, please comment it out.
\usepackage{cite}
\usepackage{amsmath,amssymb,amsfonts}
\usepackage{algorithmic}
\usepackage{graphicx}
\usepackage{textcomp}
\usepackage{xcolor}
\usepackage[utf8]{inputenc}
\usepackage{textcomp}
\usepackage{caption}

\def\BibTeX{{\rm B\kern-.05em{\sc i\kern-.025em b}\kern-.08em
    T\kern-.1667em\lower.7ex\hbox{E}\kern-.125emX}}
\begin{document}

\title{Design, Simulation and Analysis of a 5 GHz SRR Notch Filter Using Ansys HFSS}

\author{\IEEEauthorblockN{Basireddy Khyathi Sri}
\IEEEauthorblockA{\textit{Department of ECE} \\
\textit{IIIT Hyderabad}\\
Gachibowli, India \\
khyathisri.basireddy@students.iiit.ac.in \\
\textbf{Contribution:} Circular SRR design,\\ Quarter wave transformer}
\and
\IEEEauthorblockN{Chamarthy Madhan Sai Krishna}
\IEEEauthorblockA{\textit{Department of ECE} \\
\textit{IIIT Hyderabad}\\
Gachibowli, India \\
chamarthymadhan.k@students.iiit.ac.in\\
\textbf{Contribution:} Literature review, Report, \\ Quarter wave transformer}
\and
\IEEEauthorblockN{Priyanshi Jain}
\IEEEauthorblockA{\textit{Department of ECE} \\
\textit{IIIT Hyderabad}\\
Gachibowli, India \\
priyanshi.jain@research.iiit.ac.in\\
\textbf{Contribution:} --}
\and
% \hspace{2cm}
\IEEEauthorblockN{Sanjana Sheela}
\IEEEauthorblockA{\textit{Department of ECE} \\
\textit{IIIT Hyderabad}\\
Gachibowli, India \\
sanjana.sheela@students.iiit.ac.in\\
\textbf{Contribution:} Literature review, Electric coupling,\\ Square SRR}
\and
\IEEEauthorblockN{Snigdha}
\IEEEauthorblockA{\textit{Department of ECE} \\
\textit{IIIT Hyderabad}\\
Gachibowli, India \\
snigdha.stp@students.iiit.ac.in\\
\textbf{Contribution:} Quarter Wave Transformer,\\ Impedance Matching }
\and
}

\maketitle

\begin{abstract}
This paper presents the design and simulation of a split ring resonator (SRR) based notch filter operating at a resonant frequency of 5 GHz. The design utilizes a microstrip line coupled with an SRR to achieve a band-stop response. The electromagnetic simulation and optimization of the filter are performed using ANSYS HFSS software. The fundamental principles of SRR notch filters are discussed, and the impact of geometrical parameters on the filter's characteristics is analyzed. This work demonstrates the potential of SRRs for creating compact notch filters suitable for various microwave applications, including the suppression of unwanted signals in communication systems.
\end{abstract}

\begin{IEEEkeywords}
Split Ring Resonator (SRR), Notch filter, 5GHz, Ansys HFSS, Simulation, Design, Microstrip Line, Metamaterials, Band-stop filter, Resonant frequency
\end{IEEEkeywords}

\section{Introduction}

\subsection{Theoretical Background of SRR Notch Filters}
The Split Ring Resonator (SRR) is a type of metamaterial structure that exhibits unique electromagnetic properties, particularly at microwave frequencies. The SRR consists of a pair of concentric rings with a gap, which allows for the manipulation of electromagnetic waves. When excited at its resonant frequency, the SRR can create a band-stop response, effectively filtering out specific frequency bands while allowing others to pass through. This property makes SRRs suitable for applications in wireless communication systems, where interference mitigation is crucial.

\subsection{Motivation and Objectives}
\subsubsection{Motivation}
\textit{Problems with Traditional Filters:} 
\begin{itemize}
    \item \textit{Bulky Size:} Traditional filters based on $ \frac{\lambda}{2} $ or $ \frac{\lambda}{4} $ structures are physically large, making miniaturization difficult.
    \item \textit{Lower Selectivity:} Conventional filters may not sharply reject undesired frequencies(Q-factor).
    \item \textit{Poor Planar Integration}: Not all traditional filters are compatible with PCB or planar technologies.
\end{itemize}

\textit{Advantages of SRR-based Filters:}
\begin{itemize}
    \item \textit{Compact Size:} SRRs can be designed to be much smaller than traditional filters, making them suitable for modern compact devices.
    \item \textit{ High Selectivity:} The resonant nature of SRRs allows for sharp frequency rejection, improving filter performance.
    \item \textit{Planar Integration:} SRR structures can be easily integrated into PCB designs, facilitating mass production and cost-effectiveness.
\end{itemize}
\subsubsection{Objectives}
This project focuses on the design and simulation of a \textbf{Split Ring Resonator (SRR)} intended to operate as a \textit{notch filter} with a target resonant frequency of 5~GHz. The goal is to effectively attenuate signals around the resonant frequency while permitting others to pass through. The design requirements are summarized in the below table:

\begin{table}[ht]
\centering
\begin{tabular}{|l|l|}
\hline
\textbf{Parameter} & \textbf{Requirement} \\ \hline
Resonator Type     & Split Ring Resonator (SRR) \\ \hline
Operating Frequency & $ < $ 6 GHz \\ \hline
Resonant Frequency  & 5 GHz \\ \hline
S$_{11}$ (Input Reflection) & $<$ $-5$ dB at resonance \\ \hline
S$_{21}$ (Transmission) &$ >$ $-10$ dB at resonance (Notch filter) \\ \hline
Sensitivity         & $> $ 10\% \\ \hline
Size                & As compact as possible \\ \hline
\end{tabular}
\caption*{Design specifications for the SRR-based notch filter.}
\end{table}

\subsection{Paper Organization}
The next part of the paper is organized as follows: Section II provides a theoretical framework for understanding the electromagnetic principles behind SRRs and their coupling mechanisms. Section III outlines the design methodology, including substrate selection, topology choice, and parametric analysis. Section IV details the simulation methodology using Ansys HFSS, including model creation and parameter sweeps. Section V presents the results and analysis of the simulation, focusing on resonance frequency, quality factor, and S-parameter results. Finally, Section VI concludes the paper with a summary of findings and future work.

\section{Theoretical Framework}

\subsection{Electromagnetic Theory of SRRs}
Present Maxwell’s equations relevant to the SRR’s operation. Introduce resonance behavior with supporting math.

\subsection{Equivalent Circuit Models}
Model the SRR as an LC resonator. Derive relationships between the geometric parameters and the lumped elements.

\subsection{Coupling Mechanisms}
Differentiate electric vs. magnetic coupling with field diagrams or schematics. Explain near-field interaction with microstrip lines.

\subsection{Notch Filter Characteristics}
Introduce key parameters: center frequency, insertion loss, bandwidth, return loss, quality factor. Discuss notch sharpness and selectivity.

\subsection{Design Equations for SRRs}
Provide analytical equations relating SRR dimensions to resonant frequency. Set the basis for initial geometry selection.

\section{Design Methodology}

\subsection{Design Requirements and Specifications}
List the required performance: 5 GHz center, target bandwidth, notch depth, substrate limitations, and footprint constraints.

\subsection{Substrate Selection and Microstrip Design}
Describe chosen substrate (e.g., FR4 or Rogers), its dielectric constant, thickness, and loss tangent. Calculate 50-ohm microstrip width.

\subsection{SRR Topology Selection}
Compare various geometries (square, circular, spiral). Justify your selection based on performance and ease of fabrication.

\subsection{Parametric Analysis Framework}
Outline which parameters (gap, ring width, spacing) are to be swept in simulation. Describe your approach to optimization.

\subsection{Coupling Configuration}
Describe SRR placement relative to the microstrip. Provide geometry layout and explain expected coupling mode.

\subsection{Expected Performance}
Estimate resonant frequency using derived equations. Predict notch depth and bandwidth.

\section{Simulation Methodology}

\subsection{HFSS Simulation Environment Setup}
Describe the overall simulation environment and settings.

\subsection{Material Properties and Model Creation}
Assign dielectric and conductor properties. Detail 3D model dimensions and layers in HFSS.

\subsection{Excitation and Boundary Conditions}
Explain port settings and boundary types (e.g., radiation boundary). Validate open-space behavior.

\subsection{Advanced Simulation Techniques}
Discuss meshing strategies, adaptive refinement, and convergence tests.

\subsection{Parameter Sweep Configuration}
Explain how design variables were swept for optimization. Mention use of design sets or parameter studies.

\subsection{Post-Processing Methods}
Outline how S-parameters, return loss, and electric/magnetic fields were extracted and visualized.

\section{Results and Analysis}
\subsection{Resonance Frequency}

\subsection{Quality Factor}

\subsection{Sensitivity}

\subsection{S-Parameter Results}
Present \textit{S21} and \textit{S11} plots. Highlight the notch and overall filter performance.

\subsection{Z-Parameter Results}

\section{Conclusion}
\subsection{Summary of Findings}
This work presents the design and simulation of a 5 GHz SRR notch filter using Ansys HFSS. The filter demonstrates a compact size, high selectivity, and effective notch characteristics. The use of a double-ring circular SRR enables miniaturization, resulting in a compact filter layout suitable for modern RF front-end applications.

\subsection{Significance of the Results}
The designed SRR notch filter is a planar structure that can be easily integrated into existing RF systems. The sharp notch and strong rejection provide excellent interference mitigation for targeted frequency bands. 

\subsection{Limitations and Future Work}
The current SRR-based notch filter design has limitations, including a fixed notch frequency without tunability, reliance on idealized simulations that do not account for real-world factors like fabrication tolerances and substrate losses, and the absence of experimental validation. Future work could focus on integrating varactors or PIN diodes to enable tunable or switchable rejection, developing multi-band configurations by combining multiple SRRs, coupling the filter with wideband antennas for system-level analysis, and exploring advanced substrates to enhance miniaturization and performance.


\begin{thebibliography}{00}
\bibitem{b1} G. Eason, B. Noble, and I. N. Sneddon, ``On certain integrals of Lipschitz-Hankel type involving products of Bessel functions,'' Phil. Trans. Roy. Soc. London, vol. A247, pp. 529--551, April 1955.

\end{thebibliography}

\end{document}
